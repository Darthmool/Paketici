\documentclass[a4paper,12pt]{report}

\usepackage[OT1]{fontenc}  %%%paket za kodiranje teksta T2 cirilica, T1 latinica%%%
\usepackage[english, croatian]{babel}
\usepackage[top= 27mm, bottom=27mm, left=25mm, right=25mm]{geometry}
\usepackage[utf8]{inputenc}
\usepackage{graphicx}
\usepackage{enumerate}
\usepackage{amsmath}
\usepackage{hyperref}
\usepackage{multicol}
%\usepackage{multirow, multicol}
%\usepackage{array} 




\begin{document}

\begin{titlepage}

\newcommand{\HRule}{\rule{\linewidth}{0.5mm}} % Defines a new command for the horizontal lines, change thickness here

\center % Center everything on the page
 
%----------------------------------------------------------------------------------------
%	HEADING SECTIONS
%----------------------------------------------------------------------------------------

\textsc{\LARGE Matemati\v{c}ki fakultet}\\[1.5cm]

%\textsc{\LARGE Seminatski rad}\\[1.5cm] % Minor heading such as course title

%----------------------------------------------------------------------------------------
%	TITLE SECTION
%----------------------------------------------------------------------------------------

\HRule \\[0.4cm]
{ \huge \bfseries Re\v{s}enja zadataka sa prijemnih ispita}\\[0.4cm] % Title of your document
\HRule \\[1.5cm]
 
%----------------------------------------------------------------------------------------
%	AUTHOR SECTION
%----------------------------------------------------------------------------------------
\vfill 



% If you don't want a supervisor, uncomment the two lines below and remove the section above
\emph{Autor:}\\
Ivana \textsc{Janki\'{c}}\\[3cm] % Your name

%----------------------------------------------------------------------------------------
%	DATE SECTION
%----------------------------------------------------------------------------------------

{\large \today}\\[3cm] % Date, change the \today to a set date if you want to be precise

%----------------------------------------------------------------------------------------
%	LOGO SECTION
%----------------------------------------------------------------------------------------

%\includegraphics[width=0.7\textwidth]{Logo.png}\\[1cm] % Include a department/university logo - this will require the graphicx package
 
%----------------------------------------------------------------------------------------

\end{titlepage}
%ne diram to je sadrzaj
\tableofcontents

\newpage

%\par sdfdsfdsgfdgfd\dj{}

%\begin{figure}[!h]
%\inlcudegraphics[width=0.7\textwidth]{slika.png}
%\caption{Naziv slike}
%Referenca na \cite{lit1}
%\end{figure}
%numeracija
%\begin{enumerate}[1]
%\item bla
%\end{enumerate}

%\begin{itemize}
%\item bla
%\end{itemize}

%\section{Naslov}
%dfjksfndlksjjfdklsjf

%\subsection{Podnaslov}

%\par \textbf{podebljan tekst} \emph{nagla\v{s}en tekst}

%\par \DJ{}ura\dj{} Brankovi\'c
%\newpage

%%%%%%%%%%%%%%%%%%%%%%%%%%%%%%%%%%%%%%%%%%%%%%%%%%%%%%%%%
%Odavde pocinjes sa textom
%
%
%\par sdfdsfdsgfdgfd\dj{}
%\newpage
%%%%%%%%%%%%%%%%%%%


\section*{Uvod}
\par Skripta je radjena od poslednjeg prijemnog ispita ka ranijima, p od 2016.
\par DODATI JOS STVAARI :D

%\begin{figure}[!h]
%\begin{center}
%\includegraphics[width=0.7\textwidth]{slika1.jpg}
%\caption{NESTO}
%\end{center}
%\end{figure}
\newpage                            

\section*{Prijemni ispit iz 2016. godine}
\subsection*{Zadaci}

\begin{enumerate}[1.]
\item Kada je 25\% kante prazno, ona sadr\v{z}i 25 litara vode vi[e nego kada je 25\% kante puno. Koliko litara vode sadr\v{z}i puna kanta?
%ovako se stavkjaju ponudjeni odg u istom redu
\begin{multicols}{5}
\begin{enumerate}[A)]
\item 25 \item 33 \item 50 \item 75 \item 90
\end{enumerate}
\end{multicols}

\item Dvocifreni zavr\v{s}etak prirodnog broja $a$ je 16. Ako broj $a$ nije deljiv sa 8, tada je cifra jedinica broja $3a/4$ jednaka:
\begin{multicols}{5}
\begin{enumerate}[A)]
\item 0 \item 2 \item 5 \item 7 \item 8
\end{enumerate}
\end{multicols}

\item Koliko ima prirodnih brojeva manjih od 1000000 koji su deljivi ta\v{c}no jednim od brojeva 11 i 13?
\begin{multicols}{5}
\begin{enumerate}[A)]
\item 6993 \item 153846 \item 160839 \item 167832 \item 993006
\end{enumerate}
\end{multicols}

\item Najveci koeficijent polinoma $(2x + 1)^{10} $ jednak je:
\begin{multicols}{5}
\begin{enumerate}[A)]
\item 120 \item 11520 \item 13440 \item 15360 \item 16480
\end{enumerate}
\end{multicols} 

\item Brojevi 2, $\sqrt{6} - \sqrt{2} $ i $4 - 2\sqrt{3}$ \v{c}ine prva tri \v{c}lana:
\begin{enumerate}[A)]
\item aritmeti\v{c}kog,ali ne i geometrijskog niza 
\item geometrijskog,ali ne i arotmeti\v{c}kog niza
\item i aritmeti\v{c}kog i geomtrijskog niza
\item ni aritmeti\v{c}kog ni geomtrijskog niza
\item niza sa opstim \v{c}lanom $a_n = 4 - 2\sqrt{n} $
\end{enumerate}

\item Data je jedna\v{c}ina $(\frac{1 + ix}{1 - ix})^2 = i $, gde je $x$ realna nepoznata. Broj re\v{s}enja ove jedna\v{c}ine u intervalu $(0,1/2)$ je:
\begin{multicols}{5}
\begin{enumerate}[A)]
\item 0 \item 1 \item 2 \item 4 \item $\infty$
\end{enumerate}
\end{multicols}

\item Ako su $x_1$ i $x_2$ re\v{s}enja jedan\v{c}ine $x^2 - x + 15 = 0$, tada je $x_1^3 + x_2^3 - 2x_1^2 - x_2^2 + x_1x_2 + 2x_1 + x_2 -15 $ jednako:
\begin{multicols}{5}
\begin{enumerate}[A)]
\item 1 \item 87 \item 31 \item 16 \item -14
\end{enumerate}
\end{multicols}

\item Ako su $a$ i $b$ realni brojevi takvi da polinom  $x^4 + ax^3 -ax +b $ daje ostatak $2x+4$  pri deljenju polinomom $x^2 + 2x+ 1$, tada je $ab$ jednako:
\begin{multicols}{5}
\begin{enumerate}[A)]
\item 1 \item 2 \item 3 \item 4 \item 5
\end{enumerate}
\end{multicols}

\item Date su funkcije  $ f_1(x) = \ln \frac{1 + \sin{x}}{1 - \sin{x}}$ , $ f_2(x) = \arcsin{x} \cdot \arctan{x} $, $ f_3(x) = \sin{x} + \cos{x} $ i $f_4(x) = \frac{\ln{x^2}}{\sqrt[3]{x^2}} $. Ako sa $p$ ozna\v{c}imo broj parnih, a sa $n$ broj neparnih me\dj{}u ovim funkcija, ta\v{c}an je iskaz:
\begin{multicols}{3}
\begin{enumerate}[A)]
\item $p = 1$ i $n = 1$ \item $p = 2$ i $n = 2$ \item $p = 2$ i $n = 1$ \item $p = 1$ i $n = 2$ \item $p = 1$ i $n = 0$
\end{enumerate}
\end{multicols}



\item Za koje vrednosti realnog parametra $a$ jedna\v{c}ina $||x-3|-1|=a$ ima ta\v{c}no tri realna re\v{s}enja:
\begin{multicols}{5}
\begin{enumerate}[A)]
\item -1 \item 0 \item 1 \item 2 \item 3
\end{enumerate}
\end{multicols}


\item Funkcija $f$ je zadata sa $f(x) = \frac{m}{m} $ mm
\begin{multicols}{5}
\begin{enumerate}[A)]
\item 1 \item 2 \item 3 \item 4 \item 5
\end{enumerate}
\end{multicols}








\end{enumerate}







%\begin{figure}[!h]
%\begin{center}
%\includegraphics[width=1.2\textwidth]{2016b.jpg}
%\caption{Druga strana}
%\end{center}
%\end{figure}

%\newpage

\subsection*{Re\v{s}enje}




%\section{Kratak istorijski pregled}
%\par Kao što smo već istakli, postojanje familija asteroida prvi je primetio japanski astronom Hirajama još daleke 1918. godine. Trudićemo se da ovde damo što je moguće potpuniji pregled svega što se tokom proteklog stoleća dešavalo na ovom polju.

%\begin{figure}[!h]
%\begin{center}
%\includegraphics[width=0.4\textwidth]{slika16.jpg}
%\caption{Japanski astronom Kiotsugu Hirajama prvi je primetio postojanje familije asteroida}
%\end{center}
%\end{figure}
%\begin{itemize}
%\item familije asteroida postoje i mogu se pouzdano identifikovati
%\item familije su nastale sudarom dva asteroida.
%\item članovi familije ne predstavljaju dominantnu populaciju kod malih asteroida.
%\item u trenutku nastanka familije njeni članovi nisu izbačeni velikim brzinama.
%\item familije su značajno evoluirale u odnosu na post-sudarnu situaciju.
%\item inicijalno polje brzina teško da može biti rekonstruisano.
%\item starost familija se može odrediti.
%\item članovi familija su većinom re-akumulirani.
%\item asteroidi koji su predstavljali roditeljska tela familija nisu bili diferencirani.
%\end{itemize}

%Literatura
%\href{http://www.wikibooks.org}{Wikibooks home}
\begin{thebibliography}{9}
\bibitem{lit1} kkkk
\bibitem{lit2} kkkkk

\end{thebibliography}


\end{document}
