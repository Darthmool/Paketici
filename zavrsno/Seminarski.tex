\documentclass[a4paper,12pt]{report}

\usepackage[OT1]{fontenc}  %%%paket za kodiranje teksta T2 cirilica, T1 latinica%%%
\usepackage[english, croatian]{babel}
\usepackage[top= 27mm, bottom=27mm, left=25mm, right=25mm]{geometry}
\usepackage[utf8]{inputenc}
\usepackage{graphicx}
\usepackage{enumerate}
\usepackage{amsmath}
\usepackage{hyperref}
\usepackage{multicol}
%\usepackage{multirow, multicol}
%\usepackage{array} 




\begin{document}

\begin{titlepage}

\newcommand{\HRule}{\rule{\linewidth}{0.5mm}} % Defines a new command for the horizontal lines, change thickness here

\center % Center everything on the page
 
%----------------------------------------------------------------------------------------
%	HEADING SECTIONS
%----------------------------------------------------------------------------------------

\textsc{\LARGE Matemati\v{c}ki fakultet}\\[1.5cm]

%\textsc{\LARGE Seminatski rad}\\[1.5cm] % Minor heading such as course title

%----------------------------------------------------------------------------------------
%	TITLE SECTION
%----------------------------------------------------------------------------------------

\HRule \\[0.4cm]
{ \huge \bfseries Re\v{s}enja zadataka sa prijemnih ispita}\\[0.4cm] % Title of your document
\HRule \\[1.5cm]
 
%----------------------------------------------------------------------------------------
%	AUTHOR SECTION
%----------------------------------------------------------------------------------------
\vfill 



% If you don't want a supervisor, uncomment the two lines below and remove the section above
\emph{Autor:}\\
Ivana \textsc{Janki\'{c}}\\[3cm] % Your name

%----------------------------------------------------------------------------------------
%	DATE SECTION
%----------------------------------------------------------------------------------------

{\large \today}\\[3cm] % Date, change the \today to a set date if you want to be precise

%----------------------------------------------------------------------------------------
%	LOGO SECTION
%----------------------------------------------------------------------------------------

%\includegraphics[width=0.7\textwidth]{Logo.png}\\[1cm] % Include a department/university logo - this will require the graphicx package
 
%----------------------------------------------------------------------------------------

\end{titlepage}
%ne diram to je sadrzaj
\tableofcontents

\newpage

%\par sdfdsfdsgfdgfd\dj{}

%\begin{figure}[!h]
%\inlcudegraphics[width=0.7\textwidth]{slika.png}
%\caption{Naziv slike}
%Referenca na \cite{lit1}
%\end{figure}
%numeracija
%\begin{enumerate}[1]
%\item bla
%\end{enumerate}

%\begin{itemize}
%\item bla
%\end{itemize}

%\section{Naslov}
%dfjksfndlksjjfdklsjf

%\subsection{Podnaslov}

%\par \textbf{podebljan tekst} \emph{nagla\v{s}en tekst}

%\par \DJ{}ura\dj{} Brankovi\'c
%\newpage

%%%%%%%%%%%%%%%%%%%%%%%%%%%%%%%%%%%%%%%%%%%%%%%%%%%%%%%%%
%Odavde pocinjes sa textom
%
%
%\par sdfdsfdsgfdgfd\dj{}
%\newpage
%%%%%%%%%%%%%%%%%%%


\section*{Uvod}
\par Skripta je radjena od poslednjeg prijemnog ispita ka ranijima, p od 2016.
\par DODATI JOS STVAARI :D

%\begin{figure}[!h]
%\begin{center}
%\includegraphics[width=0.7\textwidth]{slika1.jpg}
%\caption{NESTO}
%\end{center}
%\end{figure}
\newpage                            

\section*{Prijemni ispit iz 2016. godine}
\subsection*{Zadaci}

\begin{enumerate}[1.]
\item Kada je 25\% kante prazno, ona sadr\v{z}i 25 litara vode vi[e nego kada je 25\% kante puno. Koliko litara vode sadr\v{z}i puna kanta?
%ovako se stavkjaju ponudjeni odg u istom redu
\begin{multicols}{5}
\begin{enumerate}[A)]
\item 25 \item 33 \item 50 \item 75 \item 90
\end{enumerate}
\end{multicols}

\item Dvocifreni zavr\v{s}etak prirodnog broja $a$ je 16. Ako broj $a$ nije deljiv sa 8, tada je cifra jedinica broja $3a/4$ jednaka:
\begin{multicols}{5}
\begin{enumerate}[A)]
\item 0 \item 2 \item 5 \item 7 \item 8
\end{enumerate}
\end{multicols}

\item Koliko ima prirodnih brojeva manjih od 1000000 koji su deljivi ta\v{c}no jednim od brojeva 11 i 13?
\begin{multicols}{5}
\begin{enumerate}[A)]
\item 6993 \item 153846 \item 160839 \item 167832 \item 993006
\end{enumerate}
\end{multicols}

\item Najveci koeficijent polinoma $(2x + 1)^{10} $ jednak je:
\begin{multicols}{5}
\begin{enumerate}[A)]
\item 120 \item 11520 \item 13440 \item 15360 \item 16480
\end{enumerate}
\end{multicols} 

\item Brojevi 2, $\sqrt{6} - \sqrt{2} $ i $4 - 2\sqrt{3}$ \v{c}ine prva tri \v{c}lana:
\begin{enumerate}[A)]
\item aritmeti\v{c}kog,ali ne i geometrijskog niza 
\item geometrijskog,ali ne i arotmeti\v{c}kog niza
\item i aritmeti\v{c}kog i geomtrijskog niza
\item ni aritmeti\v{c}kog ni geomtrijskog niza
\item niza sa opstim \v{c}lanom $a_n = 4 - 2\sqrt{n} $
\end{enumerate}

\item Data je jedna\v{c}ina $(\frac{1 + ix}{1 - ix})^2 = i $, gde je $x$ realna nepoznata. Broj re\v{s}enja ove jedna\v{c}ine u intervalu $(0,1/2)$ je:
\begin{multicols}{5}
\begin{enumerate}[A)]
\item 0 \item 1 \item 2 \item 4 \item $\infty$
\end{enumerate}
\end{multicols}

\item Ako su $x_1$ i $x_2$ re\v{s}enja jedan\v{c}ine $x^2 - x + 15 = 0$, tada je $x_1^3 + x_2^3 - 2x_1^2 - x_2^2 + x_1x_2 + 2x_1 + x_2 -15 $ jednako:
\begin{multicols}{5}
\begin{enumerate}[A)]
\item 1 \item 87 \item 31 \item 16 \item -14
\end{enumerate}
\end{multicols}

\item Ako su $a$ i $b$ realni brojevi takvi da polinom  $x^4 + ax^3 -ax +b $ daje ostatak $2x+4$  pri deljenju polinomom $x^2 + 2x+ 1$, tada je $ab$ jednako:
\begin{multicols}{5}
\begin{enumerate}[A)]
\item 1 \item 2 \item 3 \item 4 \item 5
\end{enumerate}
\end{multicols}

\item Date su funkcije  $ f_1(x) = \ln \frac{1 + \sin{x}}{1 - \sin{x}}$ , $ f_2(x) = \arcsin{x} \cdot \arctan{x} $, $ f_3(x) = \sin{x} + \cos{x} $ i $f_4(x) = \frac{\ln{x^2}}{\sqrt[3]{x^2}} $. Ako sa $p$ ozna\v{c}imo broj parnih, a sa $n$ broj neparnih me\dj{}u ovim funkcija, ta\v{c}an je iskaz:
\begin{multicols}{3}
\begin{enumerate}[A)]
\item $p = 1$ i $n = 1$ \item $p = 2$ i $n = 2$ \item $p = 2$ i $n = 1$ \item $p = 1$ i $n = 2$ \item $p = 1$ i $n = 0$
\end{enumerate}
\end{multicols}



\item Za koje vrednosti realnog parametra $a$ jedna\v{c}ina $||x-3|-1|=a$ ima ta\v{c}no tri realna re\v{s}enja:
\begin{multicols}{5}
\begin{enumerate}[A)]
\item -1 \item 0 \item 1 \item 2 \item 3
\end{enumerate}
\end{multicols}


\item Funkcija $f$ je zadata sa $f(x) = \frac{ax +b}{cx+d} $, gde su $a,b,c$ i $d$ realni brojevi. Ako je $f(0) = 1$,  $f(1) = 0$ i  $f(2) = 3$, koliko je  $f(3)$?
\begin{multicols}{5}
\begin{enumerate}[A)]
\item -1 \item $\frac{3}{2}$ \item 5 \item 2 \item 3
\end{enumerate}
\end{multicols}


\item Broj re\v{s}enja sistema jedna\v{c}ina 
\par $(x^{2} - 1)(2x -3y + 4z) = 0$
\par $4x + 5y +8z = -2$
\par $3x + y + 6z = 44$
\par u skupu realnih brojeva je: 
\begin{multicols}{5}
\begin{enumerate}[A)]
\item 0 \item 1 \item 2 \item 3 \item $\infty$
\end{enumerate}
\end{multicols}

\item Ako za realne brojeve $x$  i $y$ va\v{z}i $7 \cdot 3^{x} - 5 \cdot 2^{y} = 23$ i $2 \cdot 3^{x} + 3 \cdot 2^{y} = 42$, onda je zbir $x + y$ jednak:
\begin{multicols}{5}
\begin{enumerate}[A)]
\item 7 \item 2 \item 3 \item 4 \item 5
\end{enumerate}
\end{multicols}


\item Proizvod svih re\v{s}enja jedna\v{c}ine $ \log_{36} x^{2} + \log_6 (x + 5) -1 = 0$ je :
\begin{multicols}{5}
\begin{enumerate}[A)]
\item -36 \item -6 \item 1 \item 12 \item 6
\end{enumerate}
\end{multicols}


\item Broj celobrojnih re\v{s}enja nejedna\v{c}ine $\sin{x} < \lvert \cos{x} \rvert $ u intervalu $[ 0,8] $ jednak je: 
\begin{multicols}{5}
\begin{enumerate}[A)]
\item 4 \item 5 \item 6 \item 7 \item 8
\end{enumerate}
\end{multicols}

\item Ta\v{c}ke $M,N$ i $P$ su sredi\v{s}ta tri me\d{j}usobno mimoilazne ivice kocke. Ako je du\v{z}ina ivice $4cm$, povr\v{s}ina trougla $MNP$ je:
\begin{multicols}{5}
\begin{enumerate}[A)]
\item $8\sqrt{2}cm^2$ \item $\sqrt{2}cm^2$ \item $8\sqrt{3}cm^2$ \item $8cm^2$ \item $6\sqrt{3}cm^2$
\end{enumerate}
\end{multicols}

\item Oko kru\v{z}nije opisan je \v{c}etvorougao $ABCD$ povr\v{s}ine $90cm^2$. Ako je zbir du\v{z}ina naspramnih stranica $AB$ i $CD$ jednak $15cm$, du\v{z}ina polupre\v{c}nika kru\v{z}nice je : 
\begin{multicols}{5}
\begin{enumerate}[A)]
\item $6cm$ \item $5\sqrt{2}cm$ \item $6\sqrt{3}cm$ \item $3\sqrt{3}cm$ \item $3cm$
\end{enumerate}
\end{multicols}

\item Date su dve koncentri\v{c}ne kru\v{z}nice i du\v{z} $AB$ koja je tetiva kru\v{z}nice ve\'{c}eg, a tangenta na kru\v{z}nicu manjeg polipre\v{c}nika. Ako je $AB = 6$, onda je povr\v{s}ina prstena izme\d{j}u datih kru\v{z}nica jednaka: 
\begin{multicols}{5}
\begin{enumerate}[A)]
\item $12\pi$  \item $9\pi$ \item $\pi$ \item 9 \item $6\pi$
\end{enumerate}
\end{multicols}

\item Povr\v{s}ona kvadrata \v{c}ije su dve stranice na pravim $2x + y - 3 = 0$, $2x +y -8 = 0$ je :
\begin{multicols}{5}
\begin{enumerate}[A)]
\item $2\sqrt{3}$  \item 5 \item 4 \item 6 \item $3\sqrt{2}$
\end{enumerate}
\end{multicols}

\item Du\v{z}ine stranica o\v{s}trouglog trougla su $a = 60$, $ b = 52$ i $c$, a veli\v{c}ine odgovaraju\'{c}ih uglova su $\alpha,\beta  $ i $\gamma $. Ako je $\sin{\alpha } = \frac{12}{13}$, onda je $\sin{\gamma }$ jednak:
\begin{multicols}{5}
\begin{enumerate}[A)]
\item $\frac{56}{65}$  \item $\frac{56}{63}$ \item $\frac{39}{65}$ \item $\frac{39}{63}$ \item $\frac{63}{65}$
\end{enumerate}
\end{multicols}

\end{enumerate}







%\begin{figure}[!h]
%\begin{center}
%\includegraphics[width=1.2\textwidth]{2016b.jpg}
%\caption{Druga strana}
%\end{center}
%\end{figure}

\newpage

\subsection*{Re\v{s}enje}
\begin{enumerate}[1.]

\item 25\% kante je prazne je ekvivalnetno sa tim da je 75\% kante puno. Obele\v{z}imo sa $x$ zapreminu kante sa vodom. Prevedimo sada tekst zadatka u matemati\v{c}ki zapis:
\par Kada je 25\% kante prazno, ona sadr\v{z}i 25 litara vode vi\v{s}e nego kada je 25\% kante puno.
\par 75\% od $x$ umanjen sa 25 litara je isto sto i 25\% od $x$.
\par Matemati\v{c}ki : $ 0.75 \cdot x - 25 = 0.25 \cdot x$
\par Daljim re\v{s}avanjem ove jedna\v{c}ine dobijamo: $ 0.5 \cdot x = 25$. Ako obe strane pomo\v{z}imo sa brojem 2 dobijamo: $x = 50$, te je re\v{s}enja zadatka 50 litara. \textbf{Odgovor je pod C.}

\item Dvocifreni zavr\v{s}etak prirodnog broja $a$ je 16, matemati\v{c}ki zapisano $ a \equiv 16 (mod 100)$. Znamo da 8 ne sme da deli $a$, dok po tekstu zadatka zaklju\v{c}ujemo da 4 mora da deli $a$. Uradimo zadatak pe\v{s}aka, ispi\v{s}imo sve dvocifrene i trocifrene brojeve kod kojih va\v{z}i gorenja relacija. Krenu\'{c}emo od 16 do 916.
\par 16 - ne mo\v{z}e jer je deljivo sa 8.
\par 116 je kandidat za $a$.
\par 216 - ne mo\v{z}e jer je deljivo sa 8.
\par 316 je kandidat za $a$.
\par 416 - ne mo\v{z}e jer je deljivo sa 8.
\par 516 je kandidat za $a$.
\par 616 - ne mo\v{z}e jer je deljivo sa 8.
\par 716 je kandidat za $a$.
\par 816 - ne mo\v{z}e jer je deljivo sa 8.
\par 916 je kandidat za $a$.
\par Posmatrajmo sada kandidate, svi su deljivi sa 4. Podelimo ih sve sa 4 i dobijamo redom : 29, 79, 129, 179, 229. Po\v{s}to nas zanima samo cifra jedinice broja $3a/4$, ve\'{c} odavde mo\v{z}emo da zaklu\v{c}imo da \'{c}e to biti $9 * 3 (mod 10)$ to jest broj 7. \textbf{Odgovor je pod D.}

\item Brojeve koje posmatramo pripadaju skupu $ \{n\in N : n < 1 000 000 = 10^{6}\} $. Posmatrajmo skupove $A = \{ m \in N : 11 \cdot m < 10^{6}\}$, $B = \{ m \in N : 13 \cdot m < 10^{6}\}$ i $C = \{ m \in N : 11 \cdot 13 \cdot m < 10^{6}\}$. $A$ je skup svih brojeva deljivih sa 11, $B$ je skup svih brojeva deljivih sa 13 i $C$ skup svih brojeva deljivih sa 11 i 13. Po\v{s}to u skupu $A$ se nalaze brojevi deljivi i sa 13, a u $B$ se nalaze brojevi koji su deljivi i sa 11. Stoga re\v{s}enje predstavlja $\lvert A \rvert + \lvert B \rvert  - 2 \cdot \lvert C \rvert $. Gde $ \lvert A \rvert $ predstavlja kardinalnost skupa A. Izra\v{c}unajmo sada kardinalnosti skupova :
\par $ \lvert A \rvert = \lfloor \frac{10^{6}}{11}\rfloor   = 90909$  
\par  $ \lvert B \rvert = \lfloor \frac{10^{6}}{13}\rfloor   = 76923$
\par  $ \lvert C \rvert = \lfloor \frac{10^{6}}{11 \cdot 13}\rfloor = \lfloor \frac{10^{6}}{143}\rfloor  = 6993$  
\par Te je re\v{s}enje $ 90909 + 76923 - 2 \cdot 6993  = 153846$. \textbf{Odgovor je pod B.}

\item Koeficijent uz k-ti \v{c}lan se ra\v{c}una pomo\'{c}u formule $\left(\! \begin{array}{c} n \\k \end{array} \!\right) (2)^{k} 1^{n-k}$, \v{s}to je u ovom slu\v{c}aju isto \v{s}to i $\left(\! \begin{array}{c} n \\k \end{array} \!\right) (2)^{k} $. Sada tra\v{z}imo $ k \leq 10$ takvo da je gornji izraz maksimalan. Znamo da $\left(\! \begin{array}{c} n \\k \end{array} \!\right) $ daje koeficijente simetri\v{c}ne u odnosu na $\frac{n + 1}{2} $, pa uo\v{c}avamo da ce maksimalno biti za $ k \geq \lceil \frac{n + 1}{2}\rceil   = 6$. Na\d{j}imo sada koeficijente za $ 6 \leq k \geq 10$ : 
\par $\left(\! \begin{array}{c} 10 \\6 \end{array} \!\right)  \cdot 2^{6} = 210 \cdot 2^{6} = 6720 $
\par $\left(\! \begin{array}{c} 10 \\7 \end{array} \!\right)  \cdot 2^{7} = 120 \cdot 2^{7} = 15360 $
\par $\left(\! \begin{array}{c} 10 \\8 \end{array} \!\right)  \cdot 2^{8} = 45 \cdot 2^{8} =  11520$
\par $\left(\! \begin{array}{c} 10 \\9 \end{array} \!\right)  \cdot 2^{9} = 10 \cdot 2^{9} = 5120 $
\par $\left(\! \begin{array}{c} 10 \\10 \end{array} \!\right)  \cdot 2^{10} = 1 \cdot 2^{10} = 1024 $
Vidimo da je za $k = 7$ koeficijent najve\'{c} i iznosi 15360. \textbf{Odgovor je pod D.}

\item O\v{c}igledno je da niz nije aritmeti\v{c}ki te odgovori po A i C otpadaju. Proverimo da li je odgovor pod E. Da li su to \v{c}lanovi niza $a_n = 4 - 2\sqrt{n} $. Ra\v{c}unajmo $a_1 = 4-2 = 2$, $a_2 = 4 - 2\sqrt{2}$ - a ovaj broj nije me\d{j}u tra\v{z}enima. Ostalo je jo\v{s} proveriti da nije geometrijski niz. Pretpostavimo da jeste, na\d{j}imo sada koeficijent geometrijske progresije $q$ :
\par Znamo da je geomerijski niz oblika $b, bq, bq^2, bq^3$ itd. Te $q$ mo\v{z}emo na\'{c}i kada dva uzastopna \v{c}lana niza podelimo.  






\end{enumerate}



%\section{Kratak istorijski pregled}
%\par Kao što smo već istakli, postojanje familija asteroida prvi je primetio japanski astronom Hirajama još daleke 1918. godine. Trudićemo se da ovde damo što je moguće potpuniji pregled svega što se tokom proteklog stoleća dešavalo na ovom polju.

%\begin{figure}[!h]
%\begin{center}
%\includegraphics[width=0.4\textwidth]{slika16.jpg}
%\caption{Japanski astronom Kiotsugu Hirajama prvi je primetio postojanje familije asteroida}
%\end{center}
%\end{figure}
%\begin{itemize}
%\item familije asteroida postoje i mogu se pouzdano identifikovati
%\item familije su nastale sudarom dva asteroida.
%\item članovi familije ne predstavljaju dominantnu populaciju kod malih asteroida.
%\item u trenutku nastanka familije njeni članovi nisu izbačeni velikim brzinama.
%\item familije su značajno evoluirale u odnosu na post-sudarnu situaciju.
%\item inicijalno polje brzina teško da može biti rekonstruisano.
%\item starost familija se može odrediti.
%\item članovi familija su većinom re-akumulirani.
%\item asteroidi koji su predstavljali roditeljska tela familija nisu bili diferencirani.
%\end{itemize}

%Literatura
%\href{http://www.wikibooks.org}{Wikibooks home}
\begin{thebibliography}{9}
\bibitem{lit1} kkkk
\bibitem{lit2} kkkkk

\end{thebibliography}


\end{document}
